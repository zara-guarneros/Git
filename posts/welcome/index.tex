% Options for packages loaded elsewhere
\PassOptionsToPackage{unicode}{hyperref}
\PassOptionsToPackage{hyphens}{url}
\PassOptionsToPackage{dvipsnames,svgnames,x11names}{xcolor}
%
\documentclass[
  letterpaper,
  DIV=11,
  numbers=noendperiod]{scrartcl}

\usepackage{amsmath,amssymb}
\usepackage{iftex}
\ifPDFTeX
  \usepackage[T1]{fontenc}
  \usepackage[utf8]{inputenc}
  \usepackage{textcomp} % provide euro and other symbols
\else % if luatex or xetex
  \usepackage{unicode-math}
  \defaultfontfeatures{Scale=MatchLowercase}
  \defaultfontfeatures[\rmfamily]{Ligatures=TeX,Scale=1}
\fi
\usepackage{lmodern}
\ifPDFTeX\else  
    % xetex/luatex font selection
\fi
% Use upquote if available, for straight quotes in verbatim environments
\IfFileExists{upquote.sty}{\usepackage{upquote}}{}
\IfFileExists{microtype.sty}{% use microtype if available
  \usepackage[]{microtype}
  \UseMicrotypeSet[protrusion]{basicmath} % disable protrusion for tt fonts
}{}
\makeatletter
\@ifundefined{KOMAClassName}{% if non-KOMA class
  \IfFileExists{parskip.sty}{%
    \usepackage{parskip}
  }{% else
    \setlength{\parindent}{0pt}
    \setlength{\parskip}{6pt plus 2pt minus 1pt}}
}{% if KOMA class
  \KOMAoptions{parskip=half}}
\makeatother
\usepackage{xcolor}
\setlength{\emergencystretch}{3em} % prevent overfull lines
\setcounter{secnumdepth}{-\maxdimen} % remove section numbering
% Make \paragraph and \subparagraph free-standing
\ifx\paragraph\undefined\else
  \let\oldparagraph\paragraph
  \renewcommand{\paragraph}[1]{\oldparagraph{#1}\mbox{}}
\fi
\ifx\subparagraph\undefined\else
  \let\oldsubparagraph\subparagraph
  \renewcommand{\subparagraph}[1]{\oldsubparagraph{#1}\mbox{}}
\fi


\providecommand{\tightlist}{%
  \setlength{\itemsep}{0pt}\setlength{\parskip}{0pt}}\usepackage{longtable,booktabs,array}
\usepackage{calc} % for calculating minipage widths
% Correct order of tables after \paragraph or \subparagraph
\usepackage{etoolbox}
\makeatletter
\patchcmd\longtable{\par}{\if@noskipsec\mbox{}\fi\par}{}{}
\makeatother
% Allow footnotes in longtable head/foot
\IfFileExists{footnotehyper.sty}{\usepackage{footnotehyper}}{\usepackage{footnote}}
\makesavenoteenv{longtable}
\usepackage{graphicx}
\makeatletter
\def\maxwidth{\ifdim\Gin@nat@width>\linewidth\linewidth\else\Gin@nat@width\fi}
\def\maxheight{\ifdim\Gin@nat@height>\textheight\textheight\else\Gin@nat@height\fi}
\makeatother
% Scale images if necessary, so that they will not overflow the page
% margins by default, and it is still possible to overwrite the defaults
% using explicit options in \includegraphics[width, height, ...]{}
\setkeys{Gin}{width=\maxwidth,height=\maxheight,keepaspectratio}
% Set default figure placement to htbp
\makeatletter
\def\fps@figure{htbp}
\makeatother

\KOMAoption{captions}{tableheading}
\makeatletter
\@ifpackageloaded{tcolorbox}{}{\usepackage[skins,breakable]{tcolorbox}}
\@ifpackageloaded{fontawesome5}{}{\usepackage{fontawesome5}}
\definecolor{quarto-callout-color}{HTML}{909090}
\definecolor{quarto-callout-note-color}{HTML}{0758E5}
\definecolor{quarto-callout-important-color}{HTML}{CC1914}
\definecolor{quarto-callout-warning-color}{HTML}{EB9113}
\definecolor{quarto-callout-tip-color}{HTML}{00A047}
\definecolor{quarto-callout-caution-color}{HTML}{FC5300}
\definecolor{quarto-callout-color-frame}{HTML}{acacac}
\definecolor{quarto-callout-note-color-frame}{HTML}{4582ec}
\definecolor{quarto-callout-important-color-frame}{HTML}{d9534f}
\definecolor{quarto-callout-warning-color-frame}{HTML}{f0ad4e}
\definecolor{quarto-callout-tip-color-frame}{HTML}{02b875}
\definecolor{quarto-callout-caution-color-frame}{HTML}{fd7e14}
\makeatother
\makeatletter
\@ifpackageloaded{caption}{}{\usepackage{caption}}
\AtBeginDocument{%
\ifdefined\contentsname
  \renewcommand*\contentsname{Tabla de contenidos}
\else
  \newcommand\contentsname{Tabla de contenidos}
\fi
\ifdefined\listfigurename
  \renewcommand*\listfigurename{Listado de Figuras}
\else
  \newcommand\listfigurename{Listado de Figuras}
\fi
\ifdefined\listtablename
  \renewcommand*\listtablename{Listado de Tablas}
\else
  \newcommand\listtablename{Listado de Tablas}
\fi
\ifdefined\figurename
  \renewcommand*\figurename{Figura}
\else
  \newcommand\figurename{Figura}
\fi
\ifdefined\tablename
  \renewcommand*\tablename{Tabla}
\else
  \newcommand\tablename{Tabla}
\fi
}
\@ifpackageloaded{float}{}{\usepackage{float}}
\floatstyle{ruled}
\@ifundefined{c@chapter}{\newfloat{codelisting}{h}{lop}}{\newfloat{codelisting}{h}{lop}[chapter]}
\floatname{codelisting}{Listado}
\newcommand*\listoflistings{\listof{codelisting}{Listado de Listados}}
\usepackage{amsthm}
\theoremstyle{definition}
\newtheorem{definition}{Definición}[section]
\theoremstyle{remark}
\AtBeginDocument{\renewcommand*{\proofname}{Prueba}}
\newtheorem*{remark}{Observación}
\newtheorem*{solution}{Solución}
\newtheorem{refremark}{Observación}[section]
\newtheorem{refsolution}{Solución}[section]
\makeatother
\makeatletter
\makeatother
\makeatletter
\@ifpackageloaded{caption}{}{\usepackage{caption}}
\@ifpackageloaded{subcaption}{}{\usepackage{subcaption}}
\makeatother
\ifLuaTeX
\usepackage[bidi=basic]{babel}
\else
\usepackage[bidi=default]{babel}
\fi
\babelprovide[main,import]{spanish}
% get rid of language-specific shorthands (see #6817):
\let\LanguageShortHands\languageshorthands
\def\languageshorthands#1{}
\ifLuaTeX
  \usepackage{selnolig}  % disable illegal ligatures
\fi
\usepackage{bookmark}

\IfFileExists{xurl.sty}{\usepackage{xurl}}{} % add URL line breaks if available
\urlstyle{same} % disable monospaced font for URLs
\hypersetup{
  pdftitle={Ciencia abierta},
  pdfauthor={Zaraida Guarneros},
  pdflang={es},
  colorlinks=true,
  linkcolor={blue},
  filecolor={Maroon},
  citecolor={Blue},
  urlcolor={Blue},
  pdfcreator={LaTeX via pandoc}}

\title{Ciencia abierta}
\author{Zaraida Guarneros}
\date{2024-06-30}

\begin{document}
\maketitle

\renewcommand*\contentsname{Tabla de contenidos}
{
\hypersetup{linkcolor=}
\setcounter{tocdepth}{3}
\tableofcontents
}
\listoftables
This is the first post in a Quarto blog. Welcome!

\includegraphics{thumbnail.jpg}

Since this post doesn't specify an explicit \texttt{image}, the first
image in the post will be used in the listing page of posts.

\begin{tcolorbox}[enhanced jigsaw, left=2mm, title=\textcolor{quarto-callout-note-color}{\faInfo}\hspace{0.5em}{Nota}, colbacktitle=quarto-callout-note-color!10!white, opacitybacktitle=0.6, opacityback=0, colframe=quarto-callout-note-color-frame, colback=white, rightrule=.15mm, toprule=.15mm, bottomrule=.15mm, bottomtitle=1mm, coltitle=black, breakable, toptitle=1mm, titlerule=0mm, arc=.35mm, leftrule=.75mm]

\#Ciencia ``En el contexto de los apremiantes desafíos planetarios y
socioeconómicos, las soluciones sostenibles e innovadoras requieren
esfuerzos científicos eficientes, transparentes y dinámicos, no solo de
la comunidad científica, sino de toda la sociedad.''

\end{tcolorbox}

La Ciencia Abierta es la respuesta a la demanda de varios países para
democratizar el acceso a la información y al conocimiento

\begin{tcolorbox}[enhanced jigsaw, left=2mm, title=\textcolor{quarto-callout-warning-color}{\faExclamationTriangle}\hspace{0.5em}{Advertencia}, colbacktitle=quarto-callout-warning-color!10!white, opacitybacktitle=0.6, opacityback=0, colframe=quarto-callout-warning-color-frame, colback=white, rightrule=.15mm, toprule=.15mm, bottomrule=.15mm, bottomtitle=1mm, coltitle=black, breakable, toptitle=1mm, titlerule=0mm, arc=.35mm, leftrule=.75mm]

Para asegurar que la ciencia beneficia realmente a las personas y al
planeta y no deja a nadie atrás, es necesario transformar todo el
proceso científico. La ciencia abierta es un movimiento que pretende
hacer la ciencia más abierta, accesible, eficiente, democrática y
transparente. Impulsada por los avances sin precedentes en nuestro mundo
digital, la transición hacia la ciencia abierta permite que la
información, los datos y los productos científicos sean más accesibles
(acceso abierto) y se aprovechen de manera más fiable (datos abiertos)
con la participación activa de todas las partes interesadas (apertura a
la sociedad).

\end{tcolorbox}

\begin{tcolorbox}[enhanced jigsaw, left=2mm, title=\textcolor{quarto-callout-important-color}{\faExclamation}\hspace{0.5em}{Importante}, colbacktitle=quarto-callout-important-color!10!white, opacitybacktitle=0.6, opacityback=0, colframe=quarto-callout-important-color-frame, colback=white, rightrule=.15mm, toprule=.15mm, bottomrule=.15mm, bottomtitle=1mm, coltitle=black, breakable, toptitle=1mm, titlerule=0mm, arc=.35mm, leftrule=.75mm]

Al alentar a que la ciencia esté más conectada con las necesidades de la
sociedad y promover la igualdad de oportunidades para todos
(científicos, innovadores, encargados de la formulación de políticas y
ciudadanos), la ciencia abierta puede marcar un punto de inflexión para
hacer efectivo el derecho humano a la ciencia y reducir las diferencias
en materia de ciencia, tecnología e innovación entre los países y dentro
de ellos.

\end{tcolorbox}

\begin{tcolorbox}[enhanced jigsaw, left=2mm, title=\textcolor{quarto-callout-tip-color}{\faLightbulb}\hspace{0.5em}{Tip}, colbacktitle=quarto-callout-tip-color!10!white, opacitybacktitle=0.6, opacityback=0, colframe=quarto-callout-tip-color-frame, colback=white, rightrule=.15mm, toprule=.15mm, bottomrule=.15mm, bottomtitle=1mm, coltitle=black, breakable, toptitle=1mm, titlerule=0mm, arc=.35mm, leftrule=.75mm]

¿Es usted científico, editor, encargado de política científica, o
alguien interesado en ciencia abierta?

\end{tcolorbox}

\begin{tcolorbox}[enhanced jigsaw, left=2mm, title=\textcolor{quarto-callout-caution-color}{\faFire}\hspace{0.5em}{Precaución}, colbacktitle=quarto-callout-caution-color!10!white, opacitybacktitle=0.6, opacityback=0, colframe=quarto-callout-caution-color-frame, colback=white, rightrule=.15mm, toprule=.15mm, bottomrule=.15mm, bottomtitle=1mm, coltitle=black, breakable, toptitle=1mm, titlerule=0mm, arc=.35mm, leftrule=.75mm]

Durante los dos próximos años se buscarán aportes de todas las regiones
y de todas las partes interesadas, mediante consultas en línea abiertas,
reuniones regionales y temáticas y numerosos debates sobre las
consecuencias, los beneficios y los desafíos de la ciencia abierta en
todo el mundo.

\end{tcolorbox}

\subsection{reflexion}

najdsakjfkjfd

\subsection{pregunta}

jdfjfdnvnf

\begin{definition}[]\protect\hypertarget{def-definir}{}\label{def-definir}

¿Es usted científico, editor, encargado de política científica, o
alguien interesado en ciencia abierta?

\end{definition}



\end{document}
